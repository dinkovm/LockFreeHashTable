% !TEX TS-program = pdflatex
% !TEX encoding = UTF-8 Unicode

% This is a simple template for a LaTeX document using the "article" class.
% See "book", "report", "letter" for other types of document.

\documentclass[11pt]{article} % use larger type; default would be 10pt

\usepackage[utf8]{inputenc} % set input encoding (not needed with XeLaTeX)

%%% Examples of Article customizations
% These packages are optional, depending whether you want the features they provide.
% See the LaTeX Companion or other references for full information.

%%% PAGE DIMENSIONS
\usepackage{geometry} % to change the page dimensions
\geometry{a4paper} % or letterpaper (US) or a5paper or....
% \geometry{margin=2in} % for example, change the margins to 2 inches all round
% \geometry{landscape} % set up the page for landscape
%   read geometry.pdf for detailed page layout information

\usepackage{graphicx} % support the \includegraphics command and options

% \usepackage[parfill]{parskip} % Activate to begin paragraphs with an empty line rather than an indent

%%% PACKAGES
\usepackage{booktabs} % for much better looking tables
\usepackage{array} % for better arrays (eg matrices) in maths
\usepackage{paralist} % very flexible & customisable lists (eg. enumerate/itemize, etc.)
\usepackage{verbatim} % adds environment for commenting out blocks of text & for better verbatim
\usepackage{subfig} % make it possible to include more than one captioned figure/table in a single float
% These packages are all incorporated in the memoir class to one degree or another...

%%% HEADERS & FOOTERS
\usepackage{fancyhdr} % This should be set AFTER setting up the page geometry
\pagestyle{fancy} % options: empty , plain , fancy
\renewcommand{\headrulewidth}{0pt} % customise the layout...
\lhead{}\chead{}\rhead{}
\lfoot{}\cfoot{\thepage}\rfoot{}

%%% SECTION TITLE APPEARANCE
\usepackage{sectsty}
\allsectionsfont{\sffamily\mdseries\upshape} % (See the fntguide.pdf for font help)
% (This matches ConTeXt defaults)

%%% ToC (table of contents) APPEARANCE
\usepackage[nottoc,notlof,notlot]{tocbibind} % Put the bibliography in the ToC
\usepackage[titles,subfigure]{tocloft} % Alter the style of the Table of Contents
\renewcommand{\cftsecfont}{\rmfamily\mdseries\upshape}
\renewcommand{\cftsecpagefont}{\rmfamily\mdseries\upshape} % No bold!

%%% END Article customizations

%%% The "real" document content comes below...

\title{Dynamically Resizing Lock Free Hash Tables}
\author{Trevor Absher, Martin Dinkov}
%\date{} % Activate to display a given date or no date (if empty),
         % otherwise the current date is printed 

\begin{document}
\maketitle

\section{Abstract}

This paper expands upon the algorithm presented in \textit{Dynamic-Sized Nonblocking Hash Tables} by Liu et. al [1]. A specialization of their algorithm lied in "resizing in both directions: shrinking and growing." with the core of the algorithm being a "freezeable set abstraction, which greatly simplifies the task of moving elements among buckets during resize." [Liu et. al 2014]. Their algorithm was unique because the previous paper to attempt a lock-free hash table by Shalev and Shavit [2]  did not support a shrinking method on their hash-table. Furthermore, the freezable set abstraction in this paper expands on the normal operations of a set by adding a freeze method which prevents any thread from editing a given bucket while it is resized, simplifying the resize process [1].

In this paper, we focus on the lock-free version of the hash table, as opposed to the wait-free version. Our implementation is designed to make sure at least one thread is guaranteed to complete the method it has called. We initially built the algorithm using the pseudocode the authors provided to make an initial comparison of performance of our interpretation versus theirs. Then, we experimented on the algorithm by modifying the set in the algorithm from a single flat array to a linked-list application, endeavoring to make a more memory-efficient version at the possible cost of runtime, with the ultimate goal of comparing the benefits of and the disadvantages of this adaptation and coming up with a conclusion as to why the authors chose the implementation they did.

After our interpretation of the lock-free algorithm presented by Liu et. al was complete, we performed two different versions of tests on it. The first was a test of the data structure's concurrent performance, focusing on performance compared to the original algorithm, progress guarantees of our implementation, and the correctness condition of our implementation. Further data on these tests can be found in Section 5 - Performance Evaluation. For the second test we converted our concurrent data structure into a transactional data structure, using the open-source RSTM library. The performance of the transactional data structure was then compared to the initial performance of our adapation of the concurrent data structure. The data for these tests can be found in Section 5 - Performance Evaluation. 


\section{Categories and Subject Descriptors}

D.1.3 [\textbf{Programming Techniques}]: Concurrent Programming -- \textit{Parallel Programming;} E.2 [\textbf{Data Storage Representations}]: Hash-table Representation

\section{Outline}



\end{document}
